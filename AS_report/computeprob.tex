\subsection{SLAM Heavy Load}

The complexity of the algorithm~\ref{alg:iterSlam} for an iteration is $\sim\mathcal{O}(n.(nbDiv))$ where $n$ is the number of iteration passed and $nbDiv$ the maximum number of division of the estimate of the beacons.

Therefore a such an algorithm is not practical for an online slam. The use of the precedent algorithm  generates increasing slow-downs over the time.

To prevent those slow-downs the number of iterations can be limited by using the algorithm~\ref{alg:iterSlam} on the last n\_chosen iterations. It induces a loss of information but it is absorbable.

To cut again the load, the propagation constraint can be done with a reduce rate, not following the incoming sensor data. But the rate cannot be too reduce be can then measure of distance would be forgotten without being processed.

\subsection{Message Delay}\label{ssec:delaycompProb}

A message may be received but with ROS, it can have a important delay depending on the network and CPU load. Thus when receiving a distance to a beacon the robot can have travel a few decimetre and then measure of distance will be wrong when computed.

To address this issue the interval of the measure is inflated of the distance travelled by the robots between the measure and the computation. 